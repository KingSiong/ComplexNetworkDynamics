\documentclass{article}

\usepackage[utf8]{inputenc}
\usepackage{ctex}
\usepackage{assignpkg}
\usepackage{xeCJK}
\usepackage{amsmath, amsthm, amssymb}
\usepackage{listings,xcolor}
\usepackage{geometry} % 设置页边距
\usepackage{fontspec}
\usepackage{graphicx}
\usepackage[colorlinks]{hyperref}
\usepackage{setspace}
\usepackage{fancyhdr} % 自定义页眉页脚
\usepackage{enumerate}
\usepackage{ulem}
\usepackage{scalerel}
\usepackage{stackengine}
\usepackage{xcolor}
\usepackage{polynom}
\usepackage{algorithm}
% \usepackage{algorithmic}
\usepackage{algpseudocode}
\usepackage{chngcntr}
\usepackage{smartdiagram}
\newcommand\showdiv[1]{\overline{\smash{\hstretch{.5}{)}\mkern-3.2mu\hstretch{.5}{)}}#1}}
\newcommand\ph[1]{\textcolor{white}{#1}}

\counterwithin{figure}{subsection}
\counterwithin{table}{subsection}

\newtheorem*{thmm}{定理}
\newtheorem{thm}{定理}[section]
\newtheorem{definition}{定义}[section]
\newtheorem{lemma}{引理}[section]
\newtheorem{corollary}{推论}[section]
\newtheorem{prop}{命题}[section]
\newtheorem{attr}{性质}[section]
\newtheorem*{prf}{证明}
\newtheorem*{lprf}{引理证明}
\newtheorem{exm}{例}[section]
\newtheorem*{sol}{解}


\linespread{1.2}

\definecolor{dkgreen}{rgb}{0,0.6,0}
\definecolor{gray}{rgb}{0.5,0.5,0.5}
\definecolor{mauve}{rgb}{0.58,0,0.82}

\pagestyle{fancy}

\lhead{\CJKfamily{kai} Xi'an JiaoTong University} %以下分别为左中右的页眉和页脚
\chead{}

\rhead{\CJKfamily{kai} 第 \thepage 页}
\lfoot{} 
\cfoot{\thepage}
\rfoot{}
\renewcommand{\headrulewidth}{0.4pt} 
\renewcommand{\footrulewidth}{0.4pt}
%\geometry{left=2.5cm,right=3cm,top=2.5cm,bottom=2.5cm} % 页边距
\geometry{left=3.18cm,right=3.18cm,top=2.54cm,bottom=2.54cm}
\setlength{\columnsep}{30pt}

\renewcommand{\algorithmicrequire}{ \textbf{Input:}} %Use Input in the format of Algorithm
\renewcommand{\algorithmicensure}{ \textbf{Output:}} %UseOutput in the format of Algorithm

\makeatletter

\makeatother

\lstset{
    language    = c++,
    numbers     = left,
    numberstyle={                               % 设置行号格式
        \small
        \color{black}
        % \fontspec{Consolas}
    },
	commentstyle = \color[RGB]{0,128,0}\bfseries, %代码注释的颜色
	keywordstyle={                              % 设置关键字格式
        \color[RGB]{40,40,255}
        % \fontspec{Consolas Bold}
        \bfseries
    },
	stringstyle={                               % 设置字符串格式
        \color[RGB]{128,0,0}
        % \fontspec{Consolas}
        \bfseries
    },
	basicstyle={                                % 设置代码格式
        % \fontspec{Consolas}
        \small\ttfamily
    },
	emphstyle=\color[RGB]{112,64,160},          % 设置强调字格式
    breaklines=true,                            % 设置自动换行
    tabsize     = 4,
    frame       = single,%主题
    columns     = fullflexible,
    rulesepcolor = \color{red!20!green!20!blue!20}, %设置边框的颜色
    showstringspaces = false, %不显示代码字符串中间的空格标记
	escapeinside={\%*}{*)},
}


% \studentIds{计试91 施劲松}{2193512032}
% \studentNames{姓名}{学号}

\assignmentName{复杂网络动力学基础}
\assignmentNumber{2}
\subTitle{}

\date{10/23/2022}

\begin{document}

\makecover

\section{问题 1}

\subsection{问题描述}

试求 ER 随机网络 $G_{N, p}$ 在 $N=3000$,并在:
\begin{enumerate}
    \item $c=0.5,z=1.0$
    \item $c=1.0,z=1.0$
    \item $c=1.0,z=1.5$
    \item $c=10,z=1.0$
\end{enumerate}
四种情况下的谱密度图,给出实验过程、结果及相关分析。

\subsection{实验原理}

随机网络是由一些节点通过随机连接而组成的一种复杂网络。其构成有两种等价方法:
\begin{enumerate}
    \item ER 模型
    \item 二项式模型
\end{enumerate}
其中 ER 模型为给定 $N$ 个节点,最多可存在 $N(N-1)/2$ 条边,从这些边中随机选择 $M$ 条边就可以得到一个随机网络,它一共可以产生 $C_{N(N-1)/2}^{M}$ 种可能的随机网络,且产生每种随机网络的概率均相同。

随机网络中节点的度分布是遵循 Possion 分布的,有平均度满足:
$$
<k>=p(N-1)\approx pN \quad (N\to\infty)
$$
准确来说,对于某个节点 $v_i$,其度分布 $K$ 满足二项分布,即:
$$
K\thicksim B(N-1,p)
$$

网络频谱是其邻接矩阵 $\mathrm{A}$ 的特征值的集合。一个有 $N$ 个节点的网络必有 $N$ 个特征值 $\lambda_j$,从而定义其谱密度为:
$$
\rho(\lambda)=\frac{1}{N}\sum_{j=1}^{N}\delta(\lambda -\lambda_j)
$$
其中,
$$
\delta(x)=
\begin{cases}
    +\infty,x=0\\
    0,x\neq 0
\end{cases},
\int_{-\infty}^{+\infty}\delta(x){\mathrm{d}}x=1
$$

而对于连接概率为 $p(N)=cN^{-z}$ 的随机网络 $G_{N, p}$ 的特征谱,其平均度为 $<k>=Np=cN^{1-z}$。当 $0\leq z<1$ 时,随机网络中将出现无限聚类体,且当 $N\to\infty$ 时,$<k>\to\infty$。此时,随机网络的频谱密度呈现半圆形分布,并有:
$$
\rho(\lambda)=
\begin{cases}
    \frac{\sqrt{4Np(1-p)-\lambda^2}}{2\pi Np(1-p)},\vert \lambda\vert<2\sqrt{Np(1-p)}\\
    0, otherwise
\end{cases}
$$

\subsection{求解过程}

利用 \lstinline{python} 随机生成 ER 网络进行采样,而后获取邻接矩阵的特征值,计算谱密度。由于采用 Monte-Carlo 的随机采样,为了使不用的采样具有可加性,将谱密度绘成直方图,使得每个特征值都能落向唯一一个区间,方便统计绘图。循环采样,其流程图表示如下:

\begin{center}
    \smartdiagram[circular diagram:clockwise]{设置参数,
        多次采样生成随机ER网络, 获取随机图的谱密度, 统计绘图}
\end{center}

\subsection{实验结果与分析}

实验得到的结果图 \ref{fig:task1}。可以看到:
\begin{itemize}
    \item 当 $z>1$ 时,谱密度图度偏离半圆形结构,这是由于当 $N\to\infty$ 时,$<k>\to 0$,此时 $\rho(\lambda)$ 的奇数阶矩 ${\mathrm{M}}_{p}=0$,这意味着要回到原始节点的路径只能是沿着来时的相同节点返回,网络具有树状结构;
    \item 当 $z=1,c\leq 1$ 时,图像仍不为半圆形,这是因为当 $N\to\infty$ 时,节点平均度 $<k>\approx c$,且 $c\leq 1$,那么网络基本也为树状结构;
    \item 当 $z<1,c>1$ 时,谱密度的奇数矩 ${\mathrm{M}}_{p} \gg 0$,网络结构则发生了显著变化,出现``环''与``分支'',几乎任何结点都属于无限聚类体,谱密度呈现半圆形分布。
\end{itemize}

\begin{figure}[ht]
    \label{fig:task1}
    \centering 
    \includegraphics[width=.7\textwidth]{../task1.jpg}
    \caption{问题 1:谱密度图}
\end{figure}

本实验还绘出了 $z<1,c>1,N\to\infty$ 时的谱密度理论分布曲线($\rho'=\frac{\sqrt{4-\lambda'^2}}{2\pi}(\rho'=\sqrt{Np(1-p)\rho}, \lambda'=\lambda/\sqrt{Np(1-p)})$,即图 \ref{fig:task1} 中黑色曲线)可见其与 $z=1.5,c=10$ 图像(图 \ref{fig:task1} 中黄色曲线)基本重合,验证了分析的正确性。

\section{问题 2}

\subsection{问题描述}

试绘制WS小世界网络的度分布 $P(k)$ 图。初始网络为规则网络,选取最近邻耦合网络,其中,节点总数$N=1000$,耦合数$K=6$,$n$为未重连的边数,$\widetilde{n}$ 为重连的边数,随机化重连概率分别为 $p=0,0.1,0.2,0.4,0.6,0.9,1.0$。理论近似计算公式如下:
$$
P(k)\approx \sum_{n=0}^{\min\{k-K/2,K/2\}}C_{K/2}^{n}(1-p)^{n}p^{K/2-n}\frac{(pK/2)^{k-K/2-n}}{(k-K/2-n)!}e^{-pK/2}
$$

试分析度分布公式$P(k)$, 它的未重连边数 $n$ 的求和区间如果选取 $\max(k-K,\frac{K}{2})$,
结果又如何?当 $k\geq K/2$ 时,$k$ 最大值可为多少?请给出实验过程、结果及相关分
析。要求每个节点的度值 $k\geq K/2$ 且保证节点的连通性不被破坏。

\subsection{实验原理}

小世界的概念,简单地说就是用来描述这样一个事实:尽管一些网络系统具有很大的尺寸,但其中任意两个节点之间却有一个相对较小的距离。小世界模型是介于规则网络和随机网络之间的网络。WS 小世界模型基于两个人(Watts 和 Strogatz)的假设,模型从一个完全的规则网络出发,以一定的概率将网络中的连接打乱重连。具体的构造方法如下:
\begin{enumerate}
    \item 构造规则图。例如考虑一个含有 $N$ 个节点的最近邻耦合网络,它们围成一个环,其中每个节点都与它左右两侧的各 $K/2$ 个节点相连,耦合数 $K$ 是偶数;
    \item 随机化重连。将上面规则图中的每条边以概率 $p$ 随机地重新连接,即:将边的一个端点保持不变,而另一个端点以概 $p$ 与网络的其余 $(N-K-1)$ 个节点随机连接。其中规定:任意两个不同的节点之间至多只能有一条边(无重边)。
\end{enumerate}
最后得到的网络称为WS模型网络。

度分布公式如下:
\begin{align}
    P(k) = \sum_{n=0}^{\min(k-K/2,K/2)} C_{K/2}^{n} \binom{k-K/2-n}{k/2} p^{k-2n}
    (1-p)^{K+2n-k}
\end{align}

注意到当$N\to\infty$时,可以用泊松分布近似:
\begin{align}
    P(k) = \sum_{n=0}^{\min(k-K/2,K/2)} C_{K/2}^{n} (1-p)^n p^{K/2-n} \frac{{(pK/2)}^{k-K/2-n}}{(k-K/2-n)!} e^{-pK/2}
    \label{sg}
\end{align}

\subsection{求解过程}

本实验使用随机采样的方式完成,并用度分布公式进行验证。循环采样,流程图如下:
\begin{center}
    \smartdiagram[circular diagram:clockwise]{设置参数,
        多次采样生成随机WS网络, 获取随机图的度分布, 利用泊松分布近似公式计算理论值, 统计绘图}
\end{center}

\subsection{实验结果与分析}

实验结果为图 \ref{fig:task2}。与实验指导书给出的几乎相同。另外图中对应颜色的 $\triangle$ 为利用公式计算的结果,可以发现与随机模拟得到的曲线几乎无异,这就验证了实验及公式的正确性。
\begin{figure}[ht]
    \label{fig:task2}
    \centering
    \includegraphics[width=.7\textwidth]{../task2.jpg}
    \caption{问题 2:WS 模型度分布图}
\end{figure}

思考题:
\begin{enumerate}
    \item 它的未重连边数 $n$ 的求和区间如果选取 $\max(k-K,\frac{K}{2})$,结果又如何?\\
    在代码中调整参数绘制图 \ref{fig:task2_fake},可见只有在 $k<K/2$ 时有变化,这是因为当 $k<K/2$ 时,$P(k)$ 计算时只有 $n=0$ 时有贡献,所以在 $k<K/2$ 有值。而注意到当 $n>K/2$ 时 $C_{K/2}^{n}=0$,所以当 $n>K/2$ 无贡献,同理在 $n>k-K/2$ 时亦无贡献,故:$n\leq\min\{k-K/2,K/2\}$,更改上界并无意义。
    \begin{figure}[ht]
        \label{fig:task2_fake}
        \centering
        \includegraphics[width=0.7\textwidth]{../task2_fake.jpg}
        \caption{思考题 1:调整上界重绘度分布}
    \end{figure}
    \item 当 $k\geq K/2$ 时,$k$ 最大值可为多少?\\
    答:为 $N-1$,不过此发生的概率极小。
\end{enumerate}

\newpage

\section{感想与建议}

杨小国天下第一!

\section{源码}

问题 1 见:\lstinline{./src/task_1.py}。

问题 2 见:\lstinline{./src/task_2.py}。

\section{运行}

请确保安装了 \lstinline{python3} 以及依赖包 \lstinline{numpy}(可使用 conda 或 pip 安装),conda 安装命令并运行方法:
\begin{lstlisting}
$ cd src/
$ conda create -n lab1
$ conda activate lab1
$ conda install --yes --file requirements.txt
$ python3 task.py
\end{lstlisting}

另外,源码采用 utf-8 编码,若开启编辑器中文乱码,请切换至 utf-8 编码重新打开。

\end{document}